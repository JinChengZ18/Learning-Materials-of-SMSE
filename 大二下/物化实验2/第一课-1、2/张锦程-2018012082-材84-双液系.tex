\documentclass[4pt,a4papper]{article}
\usepackage{ctex}  
\usepackage{amsmath}
\usepackage{mathtools}
\usepackage{upgreek}
\pagestyle{headings}
\usepackage{graphicx}
\usepackage{subfigure}
\usepackage{indentfirst}
\usepackage{fancyhdr}
\usepackage{listings}
\usepackage{amsmath}
\usepackage{longtable}
\usepackage{geometry}
\usepackage[usenames,dvipsnames]{color}
\definecolor{lightgray}{RGB}{230,230,230}
\lstset{ language=, numbers=left, basicstyle=\small,
  keywordstyle=\color{blue}, commentstyle=\color{PineGreen},
  stringstyle=\color{red}, frame=shadowbox, breaklines=true,
  backgroundcolor=\color{lightgray},extendedchars=false }


\geometry{left=2.0cm,right=2.0cm,top=2.5cm,bottom=1.0cm}

%opening
\title{\zihao{-2}\textbf{双液系实验报告}}
\author{张锦程}
\date{\today}

\begin{document}
\maketitle
%\tableofcontents

\section{\zihao{4}实验目的}
1.用沸点仪测定在常压下环已烷—乙醇的气液平衡相图(要求测定组成和沸点)

2.掌握阿贝折射仪的使用方法(重点要求测折射率,由此可知两相组成)

\section{\zihao{4}实验原理}
两种挥发性液体组成的混合物,若该二组分的蒸气压不同,则溶液组成与其平衡气相的组成不同。此外,沸点和组成的关系有下列三种:

(a)理想液体混合物或接近理想液体混合物的双液系,其液体混合物的沸点介于两纯物质沸点之间

(b)各组分蒸气压对拉乌尔定律产生很大的负偏差,其溶液有最高恒沸点

(c)正偏差较大的,其溶液有最低恒沸点

第(b)、(c)两类溶液在最高或最低恒沸点时的气液两相组成相同,加热蒸发的结果只使气相总量增加,气液相组成及溶液沸点保持不变,这时的温度称恒沸点。三种情况下的相图可表示如下:
    \begin{figure}[htb]
        %\small
        \centering
        \includegraphics[width=15cm]{./figures/相图.png}
        \caption{相图示例} \label{fig: 1}
    \end{figure}
~\\
本实验要求测定具有最低恒沸点的环己烷—乙醇双液系的 T-x 相图。方法是:用沸点仪直接测定一系列不同组成溶液的气液平衡温度;收集少量馏出液(气相冷凝液)及吸取少量溶液(即液相),分别用阿贝折射仅测定其折射率,和事先已测定的已知组成的溶液折射率(折射率对组成的工作曲线)进行对比,得到成分。

\section{\zihao{4}实验操作}
\subsection{\zihao{4}操作步骤}
1.测定溶液的折射率:

用阿贝折射仪测定环己烷、无水乙醇以及由环已烷—乙醇组成的标准溶液的折射率,作折射率对组成的工作曲线

2.检查待测样品的浓度:

在加热之前,检查待测样品的浓度是否合适。若浓度不符合要求,则加环己烷或乙醇调节

3.测定液相和气相组成:

测质量百分数为10%、30%、70%、92%、96%、100%的环已烷—乙醇溶液在沸点下的液、气冷凝物质的折射率

\subsection{\zihao{4}测沸点操作}
接电源,通冷却水,按要求调节调压器,加热溶液至沸腾。待其温度计上所指示的温度保持恒定后,读下该温度值,同时停止加热,并立即在小泡中取气相冷凝液,迅速测定 
其折射率,冷却液相,然后用滴管将溶液搅均后取少量液相测定其折射率。若认为数据不可靠,重复上述操作。注意:每次测量折射率后,要将折射仪的棱镜打开用希尔球吹干,以备下次测定用。

\subsection{\zihao{4}阿贝折射率使用方法}
保证镜面清洁干燥,用滴管滴加数滴试样于辅助棱镜的毛镜面上,迅速合上辅助棱镜。转动镜筒使之垂直,调节反射镜使入射光进入棱镜,同时调节目镜的焦距,使目镜中十字线清晰明亮。调节消色散补偿器使目镜中彩色光带消失。再调节读数螺旋,使明暗的界面恰好同十字线交叉处重合。从读数望远镜中读出刻度盘上的折射率数值


\section{\zihao{4}实验结果讨论}
\subsection{\zihao{-4}数据处理过程}
1.记录原始数据:

环己烷—乙醇标准溶液每种组成对应的折射率;气液两相平衡时的沸点($t$)、器外度数($n$)、辅助温度计读数($t_s$);

2.实验中处理:

由标准溶液组成与对应的折射率做组成-折射率工作曲线;由所测折射率得到实验中气相和液相的组成;由实验数据(温度可先不校正)绘制沸点—组成草图,根据图形决定补测若干点的数据
 
3.实验后处理:

由 $t, n, t_s$ 得到校正后的沸点(t');作环己烷—乙醇体系的沸点—组成图(得到相图),并求出最低恒沸点及相应的恒沸混合物的组成

\newpage

\subsection{\zihao{-4}原始数据}
\subsubsection{\zihao{5}工作曲线}

配置环己烷含量不同的若干溶液,同时测量其折射率数据如下表:
\begin{table}[h]
\begin{tabular}{|l|l|l|l|l|l|l|}
\hline
环己烷质量分数 & 0.1753 & 0.2994 & 0.4205 & 0.5505 & 0.7037 & 0.8504 \\ \hline
折射率     & 1.3691 & 1.3764 & 1.384  & 1.3997 & 1.4105 & 1.4209 \\ \hline
\end{tabular}
\end{table}
~\\

可作出工作曲线为 $y = 1.1043 x + 1.3535$
    \begin{figure}[htb]
        %\small
        \centering
        \includegraphics[width=12cm]{./figures/工作曲线_1.png}
        \caption{环己烷 - 乙醇工作曲线,25℃} \label{fig: 2}
    \end{figure}
~\\

\subsubsection{\zihao{5}气液两相平衡时的沸点 - 气相液相成分}
实验中可测得沸点数据和气液相的折射率,并将折射率数据由工作曲线换算为组分可得:
\begin{table}[htbp]
\begin{tabular}{|c|c|c|c|c|c|c|c|}
\hline
      & 纯环己烷   & 0.96环己烷 & 0.9环己烷 & 0.695环己烷 & 0.3环己烷 & 0.1环己烷 & 纯乙醇    \\ \hline
气相折射率 & 1.4365 & 1.4217  & 1.4185 & 1.4152   & 1.3991 & 1.3813 & 1.369  \\ \hline
液相折射率 & 1.4365 & 1.4301  & 1.4258 & 1.4153   & 1.38   & 1.3718 & 1.3691 \\ \hline
气相组成& 1.0388 & 0.8536 & 0.8135 & 0.7722 & 0.5707 & 0.3479 & 0.1940 \\ \hline
液相组成& 1.0388 & 0.9587 & 0.9049 & 0.7735 & 0.3317 & 0.2290 & 0.1952 \\ \hline
温度(℃) & 80.38  & 78.52   & 64.18  & 63.3     & 71.04  & 77.67  & 77.89  \\ \hline
\end{tabular}
\end{table}


\newpage

\subsection{\zihao{-4}相图拟合}
对取得的点进行多项式拟合得到合适的曲线,数据点和拟合图像如图所示
    \begin{figure}[htb]
        %\small
        \centering
        \includegraphics[width=15cm]{./figures/相图-f.png}
        \caption{相图拟合结果} \label{fig: 3}
    \end{figure}
~\\


观察图像可得最低恒沸点为 63.3 ℃,相应的恒沸混合物的组成为 69.5\% 环己烷。

\newpage


\section{\zihao{4}附录}
\subsection{\zihao{-4}思考题}
1.使用阿贝折射仪时要注意些什么问题?如何正确使用才能测准数据? 

使用时要注意保护棱镜,清洗时只能用擦镜纸;加试样时不可加得太多,防止样本触及镜面,阿贝折射仪不能测腐蚀性液体。

2.收集气相冷凝液的小泡 D 的体积太大,对测量有何影响?

体积太大导致在开始加热时所收集到的液体无法得到有效的置换,使得收集到的液体不完全为沸点时的气态组分,从而使得测量产生误差。

3.平衡时,气液两相温度应该不应该一样?实际是否一样?怎样防止温度的差异?

理论上液相体系应当为平衡体系,测得的为平衡相图,所以温度应当一样;但是实际过程为非平衡过程,存在加热,当升温速率高时,会导致液相的温度较高,为了防止这样的差异应在接近沸腾时控制升温速度,要等温度恒定一段时间后再进行测量。

4.沸腾之后,如何控制条件使温度稳定?

使温度稳定应通过调节电压来实现,保持适当的点压。电压不能过高,以免发生爆炸;但也不能过低,不然可能会导致喷嘴流速过低,使温度计水银球处的温度出现波动,不够稳定,影响结果



\clearpage
\end{document}